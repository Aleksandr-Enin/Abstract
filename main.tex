\documentclass{article}
\usepackage[utf8]{inputenc}
\usepackage[russian]{babel} 
\usepackage{amsmath,amssymb} 
\usepackage{fullpage}

\title{Внешние биллиарды вне правильных многоугольников}
\author{Филипп Рухович\\Московский физико-технический институт}
\date{23 апреля 2020 г.}

\begin{document}

\maketitle

Рассмотрим многоугольник $\gamma$. Из точки $p$ на плоскости проведем касательную (т.е. опорную прямую) к $\gamma$ и отразим $p$ относительно точки касания. Такое преобразование называется преобразованием внешнего биллиарда. При последовательном применении такой операции, точка может оказаться периодической (т.е. вернуться в какой-то момент в себя), апериодической (никогда не вернуться в себя), а также вырожденной (внешний биллиард можно применить конечное число раз).

В центре нашего внимания оказываются следующие, открытые в общем случае проблемы периодичности:

1. Существуют ли апериодические точки для внешнего биллиарда вне правильного $n$-угольника?

2. Верно ли, что периодические точки образуют для внешнего биллиарда вне правильного $n$-угольника множество полной меры?

Случаи $n = 3, 4, 6$ являются решеточными и тривиальными; в этих случаях, апериодической точки нет, а периодические точки, как следствие, образуют множество полной меры. В~[1] С.Табачников показал, что для случая $n = 5$ апериодическая точка существует, но периодические точки образуют множество полной меры. До недавних работ [2,3], случаи $n = 3,4,6,5$ были единственными, для которых проблемы были решены. В докладе же будет рассмотрено доказательство для случаев $n = 8, 10, 12$.

Основной идеей доказательства является обнаружение и исследование ренормализационной схемы, описывающей самоподобие периодических структур. С.Табачников обнаружил и описал такую схему для случая $n = 5$; нам же удалось обнаружить такие схемы для случаев $n = 8, 10$, а также для случая $n = 12$, в котором, в отличие от остальных случаев, доказательство базируется на точных компьютерных вычисления. Такие вычисления возможны за счет того, что координаты вершин правильного
двенадцатиугольника лежат в расширении поля рациональных чисел $\mathbb{Q}[\sqrt 3]$. Аналогичные 
расширения существуют и для случаев $n = 8 (\mathbb{Q}[\sqrt 2])$ и 5,10 ($\mathbb[\sqrt2,\sqrt{5/8 - \sqrt5/8}]$). По мнению сообщества, ренормализационные схемы существуют лишь в случаях $n = 5, 10, 8, 12$.


1. S. Tabachnikov. On the dual billiard problem. — Adv. Math, 115(2), pp.221-249, 1995.

2. F. Rukhovich. Outer billiards outside a regular octagon: periodicity of almost all orbits and existence of an aperiodic orbit. — Doklady Mathematics, 2018, Vol.98, Issue 1, pp.334-337.

3. F. Rukhovich. Outer billiards outside regular dodecagon: computer proof of periodicity of almost all orbits and existence of an aperiodic point. - arXiv:1809.03791.

\end{document}

